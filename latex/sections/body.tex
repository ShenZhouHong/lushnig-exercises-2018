\section*{Sentence set A}
\subsection*{Page 66 Exercise A4, Sentences 1 --- 10}
\noindent\textbf{ὁ χρόνος ἄξει τοὺς κακοὺς καὶ ἀδίκους πρὸς τὴν δίκην.}

Time will lead evil and injustice towards justice.

\noindent\textbf{ἡ δίκη τοὺς τῶν ἀνθρώπων βίους κρίνει.}

Justice will judge the lives of men.

\noindent\textbf{ὁ μὲν ἄδικος ἄνθρωπος σχήσει πλοῦτον, ὁ δὲ ἀγαθὸς ἕξει ἀρετὴν καὶ φίλους.}

For the good man will not wish to aquire unjust wealth

\noindent\textbf{ὁ γὰρ ἄνθρωπος ὁ ἀγαθὸς οὐ βουλήσεται ἔχειν ἄδικον πλοῦτον.}

On one hand, the unjust man will have wealth. On the other hand, the just man will have virture and love.

\noindent\textbf{τὸ γὰρ ἄδικον ἔσται ἄδικον ἀεί.}

For injustice will always be injustice.

\noindent\textbf{ἐν ὕπνῳ φαίνεται ὁ θεός.}

The God appears from sleep.

\noindent\textbf{ἀγγέλλεις πόλεμον;}

You announce war?

\noindent\textbf{ἀκουσόμεθα τοὺς λόγους τῆς σοφῆς.}

We will be a called the word, the wise.

\noindent\textbf{οἱ καλοὶ καὶ ἀγαθοὶ ἀπέθνῃσκον ὑπὸ τῶν κακῶν καὶ ἀδίκων.}

The beautiful and good will perish under the evil and unjust.

\noindent\textbf{ἐβάλλετε τοὺς ξένους τοῖς λίθοις;}

You throw the foreign/stranger from stone.

\subsection*{Page 76 Exercise B5, Sentences 1 --- 10}
\noindent\textbf{ὁ σοφὸς βούλεται τὴν ἀλήθειαν λέγειν ἀεί.}

The wise will always speak the truth

\noindent\textbf{οὗτος ὁ ἄνθρωπος οὐ μὲν σοφός ἐστι, γλώσσῃ δὲ δεινός.}

That the man on one hand is not wise, on the other hand is terrible at language

\noindent\textbf{ὁ δὲ κόσμος ἤρχετο ὑπὸ τῆσδε τῆς θεᾶς;}

For the cosmos begun under the Goddess.

\noindent\textbf{ἐκεῖνος ὁ θεὸς ἦν ὁ δεσπότης ὁ τῆς θαλάττης.}

That the God was the master of the sea.

\noindent\textbf{ὁ δεσπότης καὶ ὁ δοῦλος οὐκ ἔσονταί ποτε φίλοι. [δοῦλος slave]}

The master and the slave will not be friends.

\noindent\textbf{ὅδε ὁ ἀγαθὸς δικαστὴς οὐκ ἐλάμβανεν ἄδικα δῶρα.}

This/here the good judge, takes not unjust gifts.

\noindent\textbf{Εὐριπίδης ἦν ὁ τῆς σκηνῆς σοφός.}

Euripedes was the wise playwriter.

\noindent\textbf{οἱ ἐκ τῆς θαλάσσης εἰσὶν αἰσχροὶ καὶ ἄδικοι.}

Shameful and unjust they are, out from the Mediterranean.

\noindent\textbf{τὰ δὲ τῶν τῆς θαλάσσης θεῶν ἔργα ἐστὶ καλά.}

The Mediterranean is the god of the seas.


\noindent\textbf{Εὐριπίδης ὁ ποιητὴς ἔλεγε τάδε· ὅ τι καλὸν φίλον ἀεί. [ὅ τι that which]}

Euripides the poet said on this account: That which is beautiful is always loved.

\subsection*{Page 96 Exercise C4, Sentences 1 --- 10}
\noindent\textbf{σοφή ἐστιν. νομίζουσι τήνδε εἶναι σοφήν.}

She is wise. They think she is wise.

\noindent\textbf{οὗτος ὁ πολίτης ἐστὶ ἀγαθός. νομίζω τοῦτον τὸν πολίτην εἶναι ἀγαθόν.}

The city here is good. I think this city is good.

\noindent\textbf{ἐκεῖνος ὁ νεανίας ἐστὶ ποιητής. ἐκεῖνος ὁ νεανίας νομίζει εἶναι ποιητής. νομίζεις ἐκεῖνον τὸν νεανίαν εἶναι ποιητήν;}

That the young man is a poet. That the young man thinks toi be a p9oet. That the young man to be a poet.

\noindent\textbf{ἐλευσόμεθα εἰς τὴν νῆσον. ἐνομίζομεν ἐλεύσεσθαι εἰς τὴν νῆσον. ἔφαμεν ταύτας εἰς τάσδε τὰς νήσους ἐλεύσεσθαι.}

We will come into the island. We thought we will come into the island. We take them to the island.

\noindent\textbf{ὁ λίθος ἔχει ψυχήν. ὁ σοφὸς νομίζει τὸν λίθον ἔχειν ψυχήν. ἔφη τὸν λίθον ἔχειν ψυχήν.}

The rock holds the soul. The wise say the rock holds the soul. They say the rock holds the soul.

\noindent\textbf{οἱ ποιηταὶ ἐπαίδευον τοὺς πολίτας. ἐνόμισαν τοὺς ποιητὰς παιδεύειν τοὺς πολίτας.}

The poets taught the cityu. They said the city was taught by poets.

\noindent\textbf{οὗτος ὁ ἄνθρωπος ἤνεγκε καλὰ δῶρα τοῖς θεοῖς. νομίζετε τοῦτον τὸν ἄνθρωπον ἐνεγκεῖν καλὰ δῶρα τοῖς θεοῖς; οὗτος ὁ ἄνθρωπος ἔφη ἐνεγκεῖν καλὰ δῶρα τοῖς θεοῖς.}

The man had brought good gifts to the gods. You think here the man had brought good gifts for the gods. They say that the man had brought good gifts to the gods.

\noindent\textbf{ὁ πονηρὸς ἀπέκτεινε τοὺς φίλους. τὸν πονηρόν φαμεν ἀποκτεῖναι τοὺς φίλους.}

The toilsome kills the love. The toilsome is said to kill the love.

\noindent\textbf{ἄγει δὲ πρὸς φῶς τὴν ἀλήθειαν ὁ χρόνος. ὁ ποιητὴς ἔφη τὸν χρόνον ἄγειν πρὸς φῶς τὴν ἀλήθειαν. [φῶς, τό light]}

For time leads towards the truth. The poet says the time leads towards the truth.

\noindent\textbf{ἐλύσατο αὕτη τὸ παιδίον. ἔφασαν ταύτην λύσασθαι τὸ παιδίον.}

You released this child. They said this child was released.

\section*{Sentence set B}
\subsection*{Page 38 Exercise B3, Sentences 1 --- 10}
\noindent\textbf{ἡ μὲν εἰρήνη φέρει τὸν βίον, ὁ δὲ πόλεμος θάνατον.}

On one hand, peace brings life. On the other hand, war brings death.

\noindent\textbf{ὁ ἥλιος τοῖς ἀνθρώποις τὴν ἀρχὴν τοῦ βίου φέρει.}

The sun brings the beginning of life to men.

\noindent\textbf{ὁ πλοῦτος τὴν τοῦ ἀνθρώπου ψυχὴν λύει.}

Wealth loosens the soul of men.

\noindent\textbf{ἀνάγκη μέτρον ἔχειν. [ἀνάγκη (ἐστί) + inf.: it is necessary]}

It is nessesary to posess measure.

\noindent\textbf{τὸ παιδίον ἐθέλει παιδεύεσθαι.}

The child wishes to be taught.

\noindent\textbf{ὁ δὲ χρόνος παιδεύει τὸ παιδίον.}

For time teaches the child.

\noindent\textbf{ὁ ἄνθρωπος παιδεύεται τὸ παιδίον.}

The man has educated his child.

\noindent\textbf{τὰ παιδία εἰς τὴν νῆσον πέμπεται.}

The child is sent into the island.

\noindent\textbf{τοῖς γὰρ θεοῖς ἀνάγκη τὰ δῶρα ἄγειν.}

For it is nessesary to lead gifts to the Gods.

\noindent\textbf{οἱ μὲν ἄνθρωποι τῷ νόμῳ πείθονται· τὰ δὲ παιδία τοῖς φίλοις πείθεται}

On one hand, men obey by the means of law. On the other, children obey via means of love.

\subsection*{Page 47 Exercise A4, Sentences 1 --- 10}
\noindent\textbf{ἤγομεν τὰ δῶρα εἰς τὴν νῆσον.}

We lead the gifts into the island.

\noindent\textbf{οἱ ἄνθρωποι ἐνόμιζον τὸν ἥλιον εἶναι θεόν.}

The men have regarded the sun to be a God.

\noindent\textbf{τοὺς γὰρ φίλους παρὰ τῇ ὁδῷ ἐλείπομεν.}

For we have left the friends from the road.

\noindent\textbf{σὺν τοῖς φίλοις εἰς τὴν νῆσον ἔρχεσθαι ἐβούλοντο.}

They used to come with their friends to the island.

\noindent\textbf{οἱ μὲν ἤθελον εἰρήνην ἄγειν, οἱ δὲ ἐβουλεύοντο πόλεμον ποιεῖν. [οἱ μὲν . . . οἱ δέ some . . . others; ποιεῖν to make, inf.]}

On one hand some were wishing to lead peace, on the other hand others were wishing to make war.

\noindent\textbf{ἡ τοῦ πλούτου ὁδὸς ἔφερε θάνατον τῇ ψυχῇ.}

The way of wealth carries the death of the soul.

\noindent\textbf{ἔλεγε τοὺς τῶν θεῶν λόγους ἐν ἀνθρώποις. [ἐν among]}

He spoke the word of Gods amongst men.

\noindent\textbf{οἱ ἐν τῷ οὐρανῷ θεοὶ ἔφερον τὴν δίκην τοῖς ἀνθρώποις.}

The Gods in the Heaven had brought justice to men.

\noindent\textbf{ὁ ἄνθρωπος πόνους εἶχεν.}

The man had kept toil.

\noindent\textbf{τοὺς θεοὺς δώροις ἔπειθον.}

They persuaded the Gods by the means of gifts.

\subsection*{Page 57 Exercise B5, Sentences 1 --- 10}
\noindent\textbf{οἱ μὲν ἦσαν ἀγαθοί, οἱ δὲ κακοί.}

On one hand they were good, on the other hand they were bad.

\noindent\textbf{ἀγαθαὶ ἦτε γνώμην;}

You were good in knowledge.

\noindent\textbf{πλοῦτος ἄδικος ἔφερε τύχην κακήν.}

Unjust wealth b4rings evil fortune.

\noindent\textbf{ἀθάνατος ἡ ἀρετή.}

Immortal in virtue.

\noindent\textbf{ὁ ἄνθρωπος ὁ σοφὸς οὐκ ἐπείθετο τῷ ἀδίκῳ λόγῳ.}

The wise man is convinced to speak no evil.

\noindent\textbf{ἔργον ἐστὶ τοῦ χρηστοῦ ἀνθρώπου παύειν τὸν πόλεμον. [ἔργον ἐστί + gen. it is the business of ]}

To drive away war is the business of the good man.

\noindent\textbf{οὐκ εἶχον ἃ ἐβούλοντο.}

I do not have what they wanted.

\noindent\textbf{καλὴ γὰρ ἦν ἡ νῆσος εἰς ἣν ἤγομεν τὰ δῶρα.}

For Good was lead to the island of gifts.

\noindent\textbf{καλὸς καὶ ἀγαθὸς ὁ σοφὸς ἄνθρωπος.}

The good in intellect and virtuous man.

\noindent\textbf{ἔλεγεν ὁ σοφὸς κακά; ἡ δὲ σοφὴ ἀγαθά ἔπραττεν.}

The wise in evil speaks. For the wise in good acts.
